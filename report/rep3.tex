% REPORT FOR PROJECT2 SF2565

\documentclass[a4paper,10pt]{article}

%%Packages	%%%	%%%	%%%	%%%	%%%	%%%	%%%
\usepackage{amsmath,framed}
\usepackage[latin1]{inputenc} 	
\usepackage{listings}
\usepackage{xcolor}
\usepackage{graphicx}
\usepackage{placeins}

%%Settings	%%%	%%%	%%%	%%%	%%%	%%%	%%%
\renewcommand{\d}{\text{d}}
\newcommand{\e}{\text{e}}
\newcommand{\ve}{\mathbf}
\newcommand\numb{\addtocounter{equation}{1}\tag{\theequation}}

\setlength\fboxsep{1.2mm}
\setlength\fboxrule{0.5mm}

\lstset { %
  language=C++,
  backgroundcolor=\color{black!3},	% set backgroundcolor
  basicstyle=\footnotesize,		% basic font setting
}

%%Margins	%%%	%%%	%%%	%%%	%%%	%%%	%%%
\usepackage{geometry}
\geometry{
  a4paper,
  left=35mm,
  right=35mm,
  top=35mm,
  bottom=35mm,
}

%%Header & Footer	%%%	%%%	%%%	%%%	%%%	%%%
\usepackage{fancyhdr}
\pagestyle{fancy}
\renewcommand{\headrulewidth}{1pt}
\rhead{Hanna Hultin, Mikael Perssson}
%8509080456 mitt persnr om vi vill ha det
\lhead{P2 SF565}

\title{Project 3, SF2565}
\author{Hanna Hultin, hhultin@kth.se, 931122-2468, TTMAM2 \\ Mikael Persson, mikaepe@kth.se, 850908-0456, TTMAM2}
\begin{document}
\maketitle
\subsection*{Task 1: The Curvebase class}
In this project we use an abstract class \texttt{Curvebase} to represent curves. The class is written so that it should be easy to derive different classes from \texttt{Curvebase} to represent a wide range of different curves.

We have private virtual functions for determining the value of $x$ and $y$ as well as the derivatives with respect to $x$ and $y$ for the curve given the user parameter $p$. $p$ is supposed to take values between $a$ and $b$ which are private members of the class. We also have a private member variable called \texttt{length} which is supposed to be the arc length of the curve. 

We have public member functions for determining $x$ and $y$ given the parameter $s$ which is a scaled parameter that goes from $0$ to $1$. To determine the corresponding $x$ and $y$ we use a private member function called \texttt{newtonsolve} which uses Newton's method. To compute the integral of the arc length between two values $a$ and $b$ \texttt{newtonsolve} calls the private function \texttt{integrate}. This function uses the private member functions \texttt{iSimpson} and \texttt{i2Simpson} to compute the integral according to Project 1. 

\subsection*{Task 2: The derived classes}
From the abstract class \texttt{Curvebase} we derived the classes \texttt{xLine}, \texttt{yLine} and \texttt{fxcurve} to be able to create the boundary curves for the desired grid.

\texttt{xLine} is a class that represents curves which has a constant $y$-value. Here we set $a$ to be the initial value of $x$, $b$ the final value of $x$ and length is just set to $b-a$ since we have a straight line. We also have a private variable for the constant $y$-value called $yConst$.

We use $x$ as the used parameter so the private functions for determining the values of $x$ and $y$ and the derivatives just becomes: $x(p)=p$, $y(p)=yConst$, $dx(p)=1$ and $dy(p)=0$.

We also use that we can overwrite the functions for determining $x$ and $y$ given the parameter $s$ since in this special case these values can be computed easily. So we use that $x(s) = a+s*length$ and $y(s) = yConst$. We also use that the arc length integral between $x_1$ and $x_2$ is just $x_2-x_1$.

\texttt{yLine} is a class that represents curves which has a constant $x$-value. For \texttt{yLine} everything looks the same as for \texttt{xLine} but with $x$ and $y$ interchanged everywhere.

Lastly we have the class \texttt{fxcurve}. This is a specific class used to to represent the given lower boundary curve. Here we use $x$ as the user parameter $p$ and then we implemented that $x(p) = p$, $y(p) = f(p)$, $dx(p) = 1$ and $dy(p) = f'(p)$ where $f(\cdot)$ is given in the project description and expression for the derivative was computed analytically.  
 

\subsection*{Task 3: The Domain class}
In this task we designed the class Domain which is a general class for modelling 4-sided domains and structured grids on them. This class takes references to four objects of type \texttt{Curvebase} as arguments to the constructor. The curves represented by these \texttt{Curvebase} objects are then the four sides of the \texttt{Domain}. 


The class \texttt{Domain} has a public member function called \texttt{grid\_generation} which generates a grid according to the algebraic grid generation formula.

\subsection*{Task 4: Writing the grid to a file and plotting it in Matlab}
To be able to write the generated grid to a file we added the public function \texttt{writeFile} to our class \texttt{Domain}. This function opens a binary file and writes first all the $x$-values for the grid points and then the $y$-values for all the grid points before closing the file. 

We then opened the binary file in Matlab and plotted the grid points. 


\newpage
\subsection*{Code}
\subsubsection*{Main}
\lstinputlisting[language=C++,basicstyle=\scriptsize]{../main1.cpp}
\subsubsection*{Task 1: The Curvebase Class}
\lstinputlisting[language=C++,basicstyle=\scriptsize]{../curvebase.hpp}
\lstinputlisting[language=C++,basicstyle=\scriptsize]{../curvebase.cpp}
\subsubsection*{Task 2: The derived classes}
\lstinputlisting[language=C++,basicstyle=\scriptsize]{../xline.hpp}
\lstinputlisting[language=C++,basicstyle=\scriptsize]{../xline.cpp}
\lstinputlisting[language=C++,basicstyle=\scriptsize]{../yline.hpp}
\lstinputlisting[language=C++,basicstyle=\scriptsize]{../yline.cpp}
\lstinputlisting[language=C++,basicstyle=\scriptsize]{../fxcurve.hpp}
\lstinputlisting[language=C++,basicstyle=\scriptsize]{../fxcurve.cpp}
\subsubsection*{Task 3 and 4: The Domain Class}
\lstinputlisting[language=C++,basicstyle=\scriptsize]{../fxcurve.hpp}
\lstinputlisting[language=C++,basicstyle=\scriptsize]{../fxcurve.cpp}



\end{document}





